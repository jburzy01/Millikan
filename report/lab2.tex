\documentclass[aps, reprint,amsmath,amssymb]{revtex4-1} %APS Journal
\usepackage[T1]{fontenc}
\usepackage[utf8]{inputenc}
\usepackage{lmodern}
\usepackage{microtype}
\usepackage{graphicx}
\usepackage{siunitx}
\usepackage{bm}

\renewcommand{\vec}[1]{\boldsymbol{#1}}
\newcommand{\mat}[1]{\mathbf{#1}}
\newcommand{\uv}[1]{\vec{\hat{#1}}}
\newcommand{\x}{\vec{\hat{x}}}
\newcommand{\y}{\vec{\hat{y}}}
\newcommand{\z}{\vec{\hat{z}}}
\renewcommand{\d}{\partial}
\renewcommand{\L}{\mathcal{L}}
\renewcommand{\inf}{\infty}

\begin{document}
%----------------------------------------------------------------------
% title
%----------------------------------------------------------------------
\title{PHY64 Experiment 2: The Millikan Oil-drop Experiment}
\author{Matthew S. E. Peterson}
\author{Jackson Burzynski}
\affiliation{Department of Physics and Astronomy, Tufts University}
%\date{\today} 
\maketitle

%----------------------------------------------------------------------
% Body
%----------------------------------------------------------------------
\section{Introduction}

In 1910, American physicist Robert Milikan performed the first experiment to determine the elusive value of the electron charge. Milikan observed small oil droplets


 By observing the motion of a small oil droplet in several charge states, Milikan showed that the changes in charge from one state to the next were integer multiples of the smallest observed unit of charge.

In this experiment, we measure the electron charge using the same method that Milikan used in 1910. The main device used is a cylindrical chamber manufactured by Pasco Scientific. The bottom half of the device consists of a capacitor with plate separation set by a 7.6 mm thick spacer that serves as a droplet viewing chamber. A small hole at the top of this chamber allows for oil droplets to enter the viewing region, where they are illuminated by an LED and viewed through a microscope.

Using an atomizer, oil droplets are deposited into the upper chamber. The droplets diffuse from the upper chamber through the small hole at the top of the lower chamber and enter the viewing region. In the process 


\section{Theory}


\section{Results}


\section{Error}

\section{Conclusion}


\end{document}
